\chapter{Rails fullstack solutions}
\section{JS solution (Javascript, importmap, yarn(in vendor))}

Regular JS library can be handle with \mintinline{ruby}{importmap}~\cite{importmap-rails}. Import a javascript package 
pinned by importmap is simple as \mint{javascript}{import React from 'react'}

Extra npm assets that doesn't fit into importmaps can be installed in \mintinline{bash}{/vendor/node_modules} and 
symlinked into \mintinline{bash}{/vendor/javascript}, which in by default included in the assets pipeline.


\section{Frontend solution (Bulma, dartsass-rails)}

\mintinline{ruby}{bulma}~\cite{bulma} is install in rails through \mintinline{bash}{yarn add bulma}. Then it should be 
symlinked into \mintinline{bash}{/vendor/javascript/bulma} and imported in 
\mintinline{bash}{/app/assets/stylesheets/application.scss} by \mintinline{sass}{@use 'bulma/sass';}.

Inorder to use sass mixins in rails, \mintinline{ruby}{dartsass-rails}~\cite{dartsass-rails} is required for css 
compilation. After \mintinline{ruby}{dartsass-rails} is installed, run \mintinline{bash}{rails dartsass:install} to 
complete installation. After this is done, rails development server should be run with command 
\mintinline{bash}{./bin/dev} in order to watch and complile sass code in runtime. In production, \mintinline{bash}{rails 
assets:precompile; rails s -e production} is still available for hosting the application.

\section{Page morphing solution (turbo-rails, actioncable)}

\section{Multimedia content solution (Actiontext)}

\section{Authenticate solution (devise)}

First, install devise into rails application using
\begin{minted}{bash}
  bundle add devise
  rails g devise:install
\end{minted}

Devise module is generated with \mintinline{bash}{rails g devise users}, which will generate a model called users.
In case a non-default controller is needed for devise accounts, use
\begin{minted}{bash}
  rails g devise:controllers account
  rails g devise:views account
\end{minted}

And routing to the new controller should be set in new controller with
\begin{minted}{ruby}
# /config/routes.rb
Rails.application.routes.draw do
  devise_for :users, controllers: { sessions: "account/sessions" }
\end{minted}


\printbibliography
